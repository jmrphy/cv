%%%%%%%%%%%%%%%%%%%%%%%%%%%%%%%%%%%%%%%%%
% Medium Length Graduate Curriculum Vitae
% LaTeX Template
% Version 1.1 (9/12/12)
%
% This template has been downloaded from:
% http://www.LaTeXTemplates.com
%
% Original author:
% Rensselaer Polytechnic Institute (http://www.rpi.edu/dept/arc/training/latex/resumes/)
%
% Important note:
% This template requires the res.cls file to be in the same directory as the
% .tex file. The res.cls file provides the resume style used for structuring the
% document.
%
%%%%%%%%%%%%%%%%%%%%%%%%%%%%%%%%%%%%%%%%%

%----------------------------------------------------------------------------------------
%	PACKAGES AND OTHER DOCUMENT CONFIGURATIONS
%----------------------------------------------------------------------------------------

\documentclass[margin, 12pt]{res} % Use the res.cls style, the font size can be changed to 11pt or 12pt here

\usepackage{helvet} % Default font is the helvetica postscript font
%\usepackage{newcent} % To change the default font to the new century schoolbook postscript font uncomment this line and comment the one above
\usepackage{hyperref}
\hypersetup{
    colorlinks = true,
    linkcolor = {magenta}
}


\setlength{\textwidth}{5.1in} % Text width of the document

\begin{document}

%----------------------------------------------------------------------------------------
%	NAME AND ADDRESS SECTION
%----------------------------------------------------------------------------------------

\moveleft.6\hoffset\centerline{\large\bf Justin Murphy} % Your name at the top
 
\moveleft\hoffset\vbox{\hrule width\resumewidth height 1pt}\smallskip % Horizontal line after name; adjust line thickness by changing the '1pt'
 
\moveleft.5\hoffset\centerline{Politics and International Relations}
\moveleft.5\hoffset\centerline{University of Southampton}
\moveleft.5\hoffset\centerline{Murray Building (58) } % Your address
\moveleft.5\hoffset\centerline{Office 3049} % Your address
\moveleft.5\hoffset\centerline{Southampton, UK, SO15 2GB}
\moveleft.5\hoffset\centerline{\href{http://jmrphy.net}{jmrphy.net}}
\moveleft.5\hoffset\centerline{\href{http://twitter.com/jmrphy}{@jmrphy}}
\moveleft.5\hoffset\centerline{\href{mailto:j.murphy@soton.ac.uk}{j.murphy@soton.ac.uk}}



%----------------------------------------------------------------------------------------

\begin{resume}

%----------------------------------------------------------------------------------------
%	EDUCATION SECTION
%----------------------------------------------------------------------------------------


\section{Academic Positions}
Assistant Professor, 2013 - Present
\begin{itemize}
\item University of Southampton, Department of Politics and International Relations, United Kingdom
\end{itemize}


%----------------------------------------------------------------------------------------
%	EDUCATION SECTION
%----------------------------------------------------------------------------------------

\section{Education}
Temple University, Philadelphia, Pennsylvania, USA
\begin{itemize}
\item Ph.D., Political Science, May 2014
\item M.A., Political Science, May 2011
\item B.A. (Honors, \emph{Magna Cum Laude}), Political Science and \\
Economics, 2008
\end{itemize}

University of Michigan, Ann Arbor, Michigan, USA
\begin{itemize}
\item Inter-University Consortium for Political and Social Research (ICPSR) Summer Program in Quantitative Methods,
 2012
 \end{itemize}

\section{Dissertation}
\emph{Mass Media and the Domestic Politics of Economic Globalization} \\
\begin{itemize}
\item Abstract: This dissertation argues that the mass media have played a critical but misunderstood role in the variety of national political responses to economic globalization around the world since the 1960s. Specifically, a combination of quantitative and qualitative methods in three article-length studies demonstrates how mass media have played a variety of anti-democratic roles which have gone largely unnoticed by political scientists. The first article, "Mass Media and the Domestic Politics of Economic Globalization," argues that the mass media make welfare spending less responsive to domestic groups harmed by economic globalization. Statistical tests on state-level economic data as well as individual-level survey data are found to be consistent with this theory. The second article, "Media Ownership and the Social Construction of Economic Globalization," argues that the response of mass publics toward the global economic exposure of their country varies according to the degree of foreign ownership in the national media market. Statistical analysis of state-level media ownership data and aggregate public opinion data, combined with qualitative analyses of newspaper content before and after an increase in foreign ownership, provides mixed evidence for the theory. The third article, "Why are the Most Trade-Open Countries More Likely to Repress the Media?" argues that different components of economic globalization exert contradictory pressures on state-media relations. Statistical analysis of economic data and media freedom data combined with process-tracing in Argentina and Mexico provide evidence for the theory.

\item Committee: Christopher Wlezien (chair), Orfeo Fioretos (chair),
Roselyn Hsueh, Miguel Glatzer (external reader)
\item Fields: International Relations, Political Theory, \\
Quantitative Methods \\
\end{itemize}

%----------------------------------------------------------------------------------------
%	ARTICLE UNDER REVIEW
%----------------------------------------------------------------------------------------

\section{Article Under Review} 

``Mass Media and the Domestic Politics of Globalization," under review at American Journal of Political Science

%----------------------------------------------------------------------------------------
%	WORKING PAPERS
%----------------------------------------------------------------------------------------

\section{Working Papers} 

``Media Ownership and the Social Construction of Globalization"

``Economic Globalization, Civil War, and The Pacifying Effects of Mass Media" 

``Domestic Press Freedom and the International Shaming of Human Rights Abuse"

%----------------------------------------------------------------------------------------
%	CONFERENCE PARTICIPATION
%----------------------------------------------------------------------------------------

\section{Conference Participation} 

``Mass Media and the Domestic Politics of Globalization," presented at the Midwest Political Science Association Annual Meeting, Chicago, IL, 2013 \vspace{3 mm} \\
Northeast Political Science Association Annual Meeting, Philadelphia, PA,  2011  \vspace{3 mm} \\ 
Southern Political Science Association Annual Meeting, New Orleans, LA, 2011 \vspace{3 mm} \\
Comparative Politics Workshop, Temple University, Philadelphia, PA, 2010 \\

%----------------------------------------------------------------------------------------
%	CONFERENCE PARTICIPATION
%----------------------------------------------------------------------------------------

\section{Professional Memberships} 

American Political Science Association, Member \vspace{3 mm} \\
International Studies Association, Member
%----------------------------------------------------------------------------------------
%	ADDITIONAL TRAINING SECTION
%----------------------------------------------------------------------------------------
 
\section{Professional Training}
\begin{itemize}
\item Advanced Regression (III), David Armstrong, ICPSR, 2012
\item Maximum-Likelihood Estimation, ICPSR, Dean Lacy, ICPSR, 2012
\item Social Network Analysis, ICPSR, Ann McCranie, ICPSR, 2012
\item Time-Series Modeling, Chris Wlezien, Temple University, 2011
\item Teaching Methods, Robin Kolodny, Temple University, 2011
\item Formal Modeling, Ryan Vander Weilen, Temple University, 2011
\item Political Statistics II, Kevin Arceneaux, Temple University, 2011
\item Political Statistics I, Michael Hagen, Temple University, 2010
\end{itemize}

%----------------------------------------------------------------------------------------
%	GRANTS AND AWARDS SECTION
%----------------------------------------------------------------------------------------

\section{Grants and Awards} 

Program Scholar at the Inter-University Consortium for Political and Social Research, Department of Political Science, Temple University, 2012, \$2,300 \vspace{3 mm} \\
Graduate Student Travel and Research Award, College of Liberal Arts, Temple University, 2011 \vspace{3 mm} \\ 
Graduate Student Travel and Research Award, Department of Political Science, Temple University, 2011 \vspace{3 mm} \\
Diamond Scholar, College of Liberal Arts, Temple University, 2008, \$2,700 \\

%----------------------------------------------------------------------------------------
%	TEACHING EXPERIENCE SECTION
%----------------------------------------------------------------------------------------

\section{Teaching Experience} 

Instructor of Record, Temple University\\
\begin{itemize}
\item Introduction to Quantitative Methods, Political Science 0825 (Fall 2012, Spring 2013)
\end{itemize}
Teaching Assistant, Temple University\\
\begin{itemize}
\item Introduction to Quantitative Methods, Political Science 0825 (Fall 2011, Spring 2012)
\item Foreign Governments and Politics, Political Science 1201 (Spring 2010)
\end{itemize}
 
%----------------------------------------------------------------------------------------
%	UNIVERSITY SERVICE
%---------------------------------------------------------------------------------------- 

\section{University Service}
Instructional workshop ``A Practical Introduction to R for Political Scientists",
presented to the Department of Political Science, Temple University, 2013 \vspace{3 mm}  \\
Instructional workshop ``Pierre Bourdieu and Correspondence \\ Analysis in the Social Sciences,"
presented to the Social Science Data Library Brown-bag Luncheon, Temple University, 2010

%----------------------------------------------------------------------------------------
%	COMPUTER SKILLS SECTION
%----------------------------------------------------------------------------------------

\section{Software and\\ Programming} 
\begin{itemize}
\item R
\item Stata
\item \LaTeX 
\item SPSS
\item UCINET
\item Python
\item HTML \\
\end{itemize}


%----------------------------------------------------------------------------------------
%	REFERENCES SECTION
%----------------------------------------------------------------------------------------
\section{References}

\begin{minipage}[ht]{0.5\textwidth}

\begin{itemize}
\item Dr. Christopher Wlezien \\
Temple University \\
440 Gladfelter Hall \\
1115 Polett Walk \\
Philadelphia, PA, 19122, USA \\
wlezien@temple.edu \\
215-204-7258 \\
\end{itemize}

\begin{itemize}
\item Dr. Mark Pollack \\
Temple University \\
409 Gladfelter Hall \\
1115 Polett Walk \\
Philadelphia, PA, 19122, USA \\
mark.pollack@temple.edu \\
215-204-7782 \\
\end{itemize}

\begin{itemize}
\item Dr. Kevin Arceneaux \\
Temple University \\
453 Gladfelter Hall \\
1115 Polett Walk \\
Philadelphia, PA, 19122, USA \\
kevin.arceneaux@temple.edu \\
215-204-6950 \\
\end{itemize}

\end{minipage}
\begin{minipage}[ht]{0.5\textwidth}
\begin{itemize}
\item Dr. Orfeo Fioretos \\
Temple University \\
409 Gladfelter Hall \\
1115 Polett Walk \\
Philadelphia, PA, 19122, USA \\
fioretos@temple.edu \\
215-204-8919 \\
\end{itemize}

\begin{itemize}
\item Dr. Roselyn Hsueh \\
Temple University \\
443 Gladfelter Hall \\
1115 Polett Walk \\
Philadelphia, PA, 19122, USA \\
rhsueh@temple.edu \\
215-204-7796 \\
\end{itemize}

\begin{itemize}
\item Dr. Robin Kolodny \\
Temple University \\
437 Gladfelter Hall \\
1115 Polett Walk \\
Philadelphia, PA, 19122, USA \\
rkolodny@temple.edu \\
215-204-7709 \\
\end{itemize}

\end{minipage}

%----------------------------------------------------------------------------------------

\end{resume}
\end{document}